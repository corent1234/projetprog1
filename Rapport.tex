\documentclass{article}


\title{Rapport Projet programmation}
\author{Corentin Cornou}

\begin{document}

\maketitle
\emph{Ce rapport et ce projet sont dédiés à Emmanuel Méra sans qui rien de tout cela n'aurait pu voir le jour.}

\section{Parser et lexer}
    Si j'ai d'abord commencé par vouloir créer mon propre lexer et parser j'ai rapidement (dès la première séance) changé d'avis pour utiliser ceux proposés par \textbf{ocamllex} et \textbf{ocamlyacc} devant la difficulté  notamment de l'implémantation des strcutures de la forme \textbf{-(exp)} ou \textbf{+(exp)}. Ce choix me semble avoir été extrêmement profitable notamment du fait que la documentation offre un exemple à la fois très complet et très semblable au projet...
    
\section{Arbre syntaxique}
    La gestion de l'arbre syntaxique s'est faite sans rencontrer de problème quelconque (comme sur des roulettes).
    
\section{Écriture du code assembleur}
    L'utilisation du module x86\_64 s'est déroulé sans accroc en grande partie grâce aux conseils avisés de \textbf{Emmanuel \ensuremath\heartsuit} pour les entiers. Les ennuis ont commencé pour l'ajout des fonction spécifiques à la manipulation de flottants et le principal problème rencontré ayant été de typer correctement les fonctions dans le fichier \textbf{x86\_ 64.mli} afin que le code daigne compiler. Ce qui après maints bidouillages a fini par être reglé (prière de ne pas regarder de trop près cette partie là du code). La gestion manuelle de la pile pour les flottants aurait également pu s'avérer problèmatique sans la présence d'\textbf{Emmanuel \ensuremath\heartsuit}.
\section{Lecture depuis le terminal et gestion des fichiers}
    RAS
\section{Exemples}
    Les différents exemples permettent de bien se rendre compte des différentes fonctionnalités du langage. Les exemples 3, 4, 8 et 11 ne sont pas censé compiler. 
    
\section{On vous aime \textbf{ \ensuremath\heartsuit}}
On vous aime continuez comme ça et passez de bonnes vacances \textbf{ \ensuremath\heartsuit}
\end{document}
